% !TeX spellcheck = ca
\documentclass{article}
\usepackage[utf8]{inputenc}
\usepackage{amsmath}
\usepackage{ amssymb }
\usepackage{float}
\usepackage{subfiles}
\usepackage{listings}
\usepackage{graphicx}

%Tot això hauria d'anar en un pkg, però no sé com és fa
\newcommand*{\assignatura}[1]{\gdef\1assignatura{#1}}
\newcommand*{\grup}[1]{\gdef\3grup{#1}}
\newcommand*{\professorat}[1]{\gdef\4professorat{#1}}
\renewcommand{\title}[1]{\gdef\5title{#1}}
\renewcommand{\author}[1]{\gdef\6author{#1}}
\renewcommand{\date}[1]{\gdef\7date{#1}}
\renewcommand{\baselinestretch}{1.5}
\renewcommand{\maketitle}{ %fa el maketitle de nou
	\begin{titlepage}
		\raggedright{UNIVERSITAT DE LLEIDA \\
			Escola Politècnica Superior \\
			Grau en Enginyeria Informàtica\\
			\1assignatura\\}
		\vspace{5cm}
		\centering\huge{\5title \\}
		\vspace{3cm}
		\large{\6author} \\
		\normalsize{\3grup}
		\vfill
		Professorat : \4professorat \\
		Data : \7date
\end{titlepage}}
%Emplenar a partir d'aquí per a fer el títol : no se com es fa el package
%S'han de renombrar totes, inclús date, si un camp es deixa en blanc no apareix


\title{Solver Report}
\author{Aiax Faura Vilalta, Ian Palacín Aliana}
\date{Divendres 24 d'abril}
\assignatura{Programació avançada en intel·ligència artificial}
\professorat{Josep Argelich Roma}
\grup{}

\renewcommand{\refname}{Bibliografia}

%Comença el document
\begin{document}
	\maketitle
	\thispagestyle{empty}


%Fer intro?
\section{GSAT vs WALKSAT}
The first step we took in developing the solver was to create the basic structure
from which all the solvers we have seen were implemented.
Once we had the basic structure, we implemented a basic GSAT and polished it. Two early mistakes
that we made were:
\begin{itemize}
	\item We assumed that GSAT was faster than WALKSAT, and focussed on it.
	\item We gave a lot of weight to optimizing flow control and python code in general,
	giving less importance to the algorithm itself.
\end{itemize}
Realizing these two errors we began to investigate more about SAT and the different solvers,
and the ways in which they worked. There is when we realized that WALKSAT could be much more 
faster for the random cnfs we were generating, so we implemented it.
We also implemented tabu search, but led to worse times.  
Reaffirming the statement above, we read on the internet that WALKSAT used to evaluate fewer
possibilities and our case was no exception, it went much faster. This is because in our case,
for the generated formulas, the evaluation time of each flip in GSAT and WalkSat was very similar,
therefore, as WalkSat needed fewer evaluations was always the winner.


%Primer vam provar GSAT
%Tothom deia que era millor WALKSAT per random cnf formulas
%Vam provar WALKSAT i va anar la hostia de ràpid
% Vam fer random walks?
%Ens feia ordres de magnitud menys iteracions per trobar la solució

\section{Data structures}
The initial data structure was storing in a list the sequence of clauses as it is. We also tried to 
do it with dictionaries ant it didn't improve much.
Once everything worked properly we implemented the data structure mentioned in class, which 
lowered the timing aggressively.

We also tried different variations of the mentioned structure, like trying to sort the rows by
number of occurrences throughout the execution, or splitting them into different structures 
using dictionaries. None of these variations convinced us, they only introduced complexity 
into the system without really improve the timing, so we didn't leave any.
%Estructura de dades inicial, anava de cul
%Més tard la mencionada a classe, anava follada

%També vam provar amb diccionaris

\section{General Observations}
As we mentioned, we noticed that the most substantial improvements came form variations in the algorithm,
an important parameter was the flip settings and random probability. Depending on how they were they 
gave great or terrible results. Tweaking them was the key to came up with a configuration that finally
gave great outcomes. After seeing the results of the race, we believe that what has truly distanced 
or solver from the others is not a better implementation of the data structures, nor a better 
implementation of the algorithm, but a better configuration of flips and randoms.

Another turning point was the broken, which we reached after multiple suitors.
% Explicar broken aqui?

%Importava molt la configuració de flips i probabilitat de random, així que vam 
%provar uns quants 

%Ens en vam adonar que el broken era la clau

\section{Profiler}
To have more knowledge about our solver we used the python profiler cProfile, wich gave us 
additional information, such as the number of calls, or the time in each function. This 
way we have been able to see more clearly the bottlenecks of our implementation. For 
example, at one point of the implementation we made use of the python function \textit{deepcopy}.
After profiling it became clear to us that it was better to explore a different approach, 
as it was taking up to 40\% of the total execution time.


%Vam utlitzar un profiler per trobar colls d'ampolla

\end{document}